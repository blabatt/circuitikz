
%%%%%%%%%%%%%%%%%%%%%%%%%%%%%%%%%%%%%%%%%%%%%%%%%%%%
\section{Annotations}

%\textit{... where }\texttt{<wire\_opt>}\textit{ $\in$:}


%%%%%%%%%%%%%%%%%%%%%%%%%%%%%%%%%%%%%%%%%%
\subsection*{Path Adornments}

\textit{Bipole wire-end shorthands:}\\
\entry{45mm}{\textbackslash draw (0,0) to[short] (2,0);}{wire only}\\
\entry{45mm}{\textbackslash draw (0,0) to[open] (2,0);}{open wire}\\
\entry{45mm}{\textbackslash draw (0,0) to[R,o-o] (2,0);}{open nodes}\\
\entry{45mm}{\textbackslash draw (0,0) to[R,*-*] (2,0);}{closed}\\
\entry{45mm}{\textbackslash draw (0,0) to[R, *-o] (2,0);}{mixed ends}\\




%%%%%%%%%%%%%%%%%%%%%%%%%%%%%%%%%%%%%%%%%%
\subsection*{Labels}

\textit{An optional less-than (<) sign changes direction of a flow symbol (i, v, or f); to put a label above or below, use a carat (\textasciicircum) or underscore (\textunderscore) respectively. Use } \texttt{smartlabels}, \texttt{rotatelabels}, \textit{ or } \texttt{straightlabels}, \textit{ for control over automatic label rotations, and } \texttt{halign} \textit{ or } \texttt{valign} \textit{ for label alignment.}

\begin{itemize}
    \item i= \% current arrow
    \item v=<volts> \% volt-diff
    \item f= \% current arrow
    \item l=<name> \% label
    \item l2=<lab1> and <lab2>
\end{itemize}

\textit{Flow \& text placement \& font can be controlled:}\\
\entry{45mm}{\textbackslash ctikzset\{current/distance=.2\}}{move arrow}\\
\entry{50mm}{\textbackslash ctikzset\{voltage/label distance=.5\}}{mv label}\\
\entry{45mm}{\textbackslash ctikzset\{current/font=\textbackslash tiny\}}{font}\\

%%%%%%%%%%%%%%%%%%%%%%%%%%%%%%%%%%%%%%%%%%
\subsection*{Units}

\begin{multicols}{2}
\begin{itemize}[label={}]
    \item \underline{Normal}
    \item <\textbackslash ohm>
    \item <\textbackslash milli \textbackslash farad>
    \item <\textbackslash siemens>
    \item \underline{SIunits}
    \item \textbackslash SI \{5\}\{k\textbackslash ohm\}
    \item \textbackslash SI \{-10\}\{V\}
    \item \textbackslash SI \{8\}\{S\}
\end{itemize} 
\end{multicols}



%%%%%%%%%%%%%%%%%%%%%%%%%%%%%%%%%%%%%%%%%%
\subsection*{Anchors}
\textit{Each element and annotation defines its own set of anchors which can be referenced and moved:}\\
\resizebox{5cm}{!}{\newcommand{\marknode}[2][45]{
\node[circle, draw, red, inner sep=1pt,pin={[red, font=\tiny]#1:#2}] at (#2.center) {};
}

\begin{circuitikz}[american]

\draw (0,0) to [R=L1, a=A1, name=L1] ++(3,0) 
            to [R, l2_=L2 and 2L, a^=A2, name=L2] ++(3,0);
\marknode{L1} \marknode{L1label} \marknode[0]{L1annotation}
\marknode{L2} \marknode[0]{L2label} \marknode{L2annotation}
\draw[blue] (L2label.south west) rectangle (L2label.north east);

\draw (0,1.8) to [R=R1, v=V1, i=I1, f>^=F1, name=R1] ++(3,0) to [R, v<=V2, i^=I2, f>^=F2, name=R2] ++(3,0);

\marknode[0]{R1voltage} \marknode[0]{R2voltage} \marknode[90]{R1current} \marknode[90]{R2current} \marknode{R1flow} \marknode{R2flow}

\end{circuitikz}

}

\ 
\\
