\section{Advanced}

%%%%%%%%%%%%%%%%%%%%%%%%%%%%%%%%%%%%%%%%%%%%%%%%%%%%
\subsection*{Reusable Blocks}


\textit{Define new commands that place one or more elements and wires, using only relative (local) coords. Commands can be parameterized and blocks and sub-block elements can/should be named:}\\[2mm]
\code{\textbackslash newcommand* \{\textbackslash myblock\}[1]\{ } \% define cmd. \\
\code{node[sr-ff](\#1-FF)\{\}} \% use param. \#1 \\
%\% use only \underline{relative coordinates}: \\
\code{node[and prt](\#1-AND1) at (\$(\#1-FF) + (0,-2)\$)}
\code{\dots \}} \% end of definition \\[2mm]
\% now use the newly-defined command:\\
\code{\textbackslash draw (0,0) \textbackslash myblock[B1];} \% place block \\
\code{\textbackslash draw (0,4) \textbackslash myblock[B2];} \% and another \\[2mm]
\% can reference sub-block components:\\
\code{\textbackslash draw (B1-FF.pin 1) -{}- (B2-AND.in 2); } \\



%%%%%%%%%%%%%%%%%%%%%%%%%%%%%%%%%%%%%%%%%%%%%%%%%%%%
\subsection*{New Components}



%%%%%%%%%%%%%%%%%%%%%%%%%%%%%%%%%%%%%%%%%%%%%%%%%%%%
\subsection*{Package Version}

\href{https://tex.stackexchange.com/questions/521545/draw-flip-flop-with-circuitikz}{Upgrading on overleaf}

\ \\

